\documentclass{article}

\usepackage[T1]{fontenc}
\usepackage[utf8]{inputenc}
\usepackage[frenchb]{babel}
\usepackage{pgffor}
\usepackage{ifthen}

\usepackage{tikz}
\usetikzlibrary{matrix,fit,calc}

\usepackage{rapport}

\author{Georges Dupéron}
\title{Langage de programmation}

\begin{document}

\maketitle

\section{Objectif}

Le but de ce projet est de mettre au point un langage de programmation utilisant le paradigme du dataflow\footnote{http://en.wikipedia.org/wiki/Dataflow}, n'ayant pas de primitives fixes. 
\section{Recherche}

\subsection{Programme en dataflow}

Examinons un programme simple exprimé dans le paradigme du dataflow~:

\begin{figure}[h]
  \centering
  \begin{tikzpicture}
    \bloc[t]{b1/+}{{a}{b}{c}}{{d}{e}}
    \tikzset{b2/.style={right of=b1-out-d,matrix anchor=b2-in-x.center}}
    \bloc[t]{b2/\frac{x\times y}{z}}{{x}{y}{z}}{{t}}
    \draw (b1-out-d) -- (b2-in-x);
    \draw (b1-out-e) -- (b2-in-y);
    \draw ($ .5*(b1-out-e)+.5*(intersection of b1-out-e--b2-in-y and b2-in-y--b2-in-z) $) node[dot] {} |- (b2-in-z);
    
    \tikzset{e1/.style={left of=b1-in-a,matrix anchor=e1-out-val.center}}
    \bloc[t]{e1/1}{}{{val/}}
    \draw (e1-out-val) -- (b1-in-a);
    
    \tikzset{e2/.style={left of=b1-in-b,matrix anchor=e2-out-val.center}}
    \bloc[t]{e2/2}{}{{val/}}
    \draw (e2-out-val) -- (b1-in-b);
    
    \tikzset{e3/.style={left of=b1-in-c,matrix anchor=e3-out-val.center}}
    \bloc[t]{e3/3}{}{{val/}}
    \draw (e3-out-val) -- (b1-in-c);
    
    \draw[thick,->] (b2-out-t) -- ++(1,0);
  \end{tikzpicture}
  \caption{Un programme simple en dataflow.}
\label{fig:simple-dataflow}
\end{figure}


\end{document}


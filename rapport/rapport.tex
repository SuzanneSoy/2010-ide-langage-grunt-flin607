\documentclass{article}

\usepackage[T1]{fontenc}
\usepackage[utf8]{inputenc}
\usepackage[frenchb]{babel}
\usepackage{pgffor}
\usepackage{ifthen}

\usepackage{tikz}
\usetikzlibrary{matrix,fit}

\usepackage{rapport}

\author{Georges Dupéron}
\title{Langage de programmation}

\begin{document}

\maketitle

\section{Objectif}

Le but de ce projet est de mettre au point un langage de programmation utilisant le paradigme du dataflow\footnote{http://en.wikipedia.org/wiki/Dataflow}, n'ayant pas de primitives fixes. 
\section{Recherche}

\subsection{Programme en dataflow}

Examinons un programme simple exprimé dans le paradigme du dataflow~:

\tikzset{
  bloc/.style={
    rectangle,
    rounded corners,
    draw=black,
    very thick,
  },
  bloc-label/.style={
    bloc-label-pos,
  },
  bloc-label-pos/.style={
    color=blue,
    inner sep=0,
  },
  ports/.style={
  },
  in-ports/.style={
    ports,
    anchor=east,
  },
  out-ports/.style={
    ports,
    anchor=west,
  },
  dot/.style={
    coordinate,
    circle,
    inner sep=0pt,
    minimum size=2pt,
    fill=black,
  },
  port-dot/.style={
    dot,
  },
  port-label/.style={
    port-label-pos,
  },
  port-label-pos/.style={
    color=black!66!white,
  },
  in-port-dot/.style={
    port-dot,
  },
  in-port-label/.style={
    port-label,
  },
  in-port-label-pos/.style={
    port-label-pos,
  },
  out-port-dot/.style={
    port-dot,
  },
  out-port-label/.style={
    port-label,
  },
  out-port-label-pos/.style={
    port-label-pos,
  },
}
\shorthandoff{:} 
\begin{tikzpicture}
  \node[bloc-label] (+1) {$+$};
  \matrix[in-ports] at (+1.west) (+1In) {
    \node[in-port-label,in-port-dot] (+1A) [label=right:$A$] {};\\
    \node[in-port-label,in-port-dot] (+1B) [label=right:$B$] {};\\
    \node[in-port-label,in-port-dot] (+1C) [label=right:$C$] {};\\
  };
  \matrix[out-ports] at (+1.east) (+1Out) {
    \node[out-port-label,out-port-dot] (+1A) [label=left:$D$] {};\\
    \node[out-port-label,out-port-dot] (+1B) [label=left:$E$] {};\\
  };
 \node[bloc, fit = (+1) (+1In) (+1Out)] {};
\end{tikzpicture}
\shorthandon{:}

\begin{tikzpicture}
  \bloc[t]{b1/+}{{a}{b}{c}}{{d}{e}}
  \tikzset{b2/.style={right of=b1-out-d,matrix anchor=b2-in-x.west}}
  \bloc[t]{b2/\frac{x\times y}{z}}{{x}{y}{z}}{{t}}
  \draw (b1-out-d) -- (b2-in-x);
  \draw (b1-out-e) -- (b2-in-y);
  \draw (b1-out-e) ++(0.5,0) node[dot] {} |- (b2-in-z);
\end{tikzpicture}



\end{document}

